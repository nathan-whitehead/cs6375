%%% used to control headers and footers
\usepackage{fancyhdr}
\usepackage{extramarks}
\usepackage{parskip}
%%%

%%% used for better equations and theorems
\usepackage{amsmath}
\usepackage{amsthm}
\usepackage{amsfonts} %don't think I need this one anymore but just in case
%%%

%%%used to create graphics in latex
\usepackage{tikz}
\usetikzlibrary{automata,positioning}
%%%

%%% used for writing pseudocode if needed
\usepackage[plain]{algorithm}
\usepackage{algpseudocode}
%%%

%%% improves control over enumerate, itemize, and description.
\usepackage{enumitem}
%%%




%%%%%%%%%%%%%%%%% tables libraries %%%%%%%%%%%%%%%%%
\usepackage{tabularray}
\usepackage{siunitx}
\UseTblrLibrary{booktabs, siunitx}
\usepackage{booktabs}
\usepackage{longtable}

%used to make multicol tables.
\usepackage{multicol}
\newsavebox\ltmcbox%

%replace the \subfile{...} with your own to make them work
%\setbox\ltmcbox\vbox{
%\makeatletter\col@number\@ne
%\subfile{files/filename.tex}
%\unskip
%\unpenalty
%\unpenalty}

%used to fix extra left space when trying multicol tables
\usepackage{changepage}
% hyperref is used to make the table of contents clickable
\usepackage{hyperref}
\hypersetup{
    colorlinks=true,
    linkcolor=blue,
    filecolor=magenta,
    urlcolor=blue,
    pdfpagemode=FullScreen,
    }
\urlstyle{same}

%adjust spacing as necessary, not a great fix but at least it works
%\setlength\columnsep{-2in}
%\begin{adjustwidth}{-2.5in}{}
%%%




%%% improves word and letter spacing
\usepackage{microtype}
%%%

%%% using listings to create inserted code into the doc.
%(style code from overleaf, much better than before.
\usepackage{listings}
\usepackage{xcolor}


\definecolor{codegreen}{rgb}{0,0.6,0}
\definecolor{codegray}{rgb}{0.5,0.5,0.5}
\definecolor{codepurple}{rgb}{0.58,0,0.82}
\definecolor{backcolour}{rgb}{0.95,0.95,0.92}
\definecolor{codepink}{rgb}{0.898, 0.247, 0.451}
\definecolor{codebrown}{rgb}{0.525, 0.514, 0.439}

\usepackage{fontspec}
%\setmonofont{Monaco}

\usepackage{etoolbox}
\AtBeginEnvironment{verbatim}{\footnotesize}

\lstdefinestyle{mystyle}{
   backgroundcolor=\color{backcolour},
   commentstyle=\color{codebrown},
   keywordstyle=\color{codepink},
   numberstyle=\tiny\color{codegray},
   stringstyle=\color{codegreen},
   basicstyle=\ttfamily\footnotesize,
   breakatwhitespace=false,         
   breaklines=true,                 
   captionpos=b,                    
   keepspaces=true,                 
   numbers=left,                    
   numbersep=5pt,                  
   showspaces=false,                
   showstringspaces=false,
   showtabs=false,          
   tabsize=2
}

\lstset{style=mystyle}

% Define CSS as a custom language
\lstdefinelanguage{CSS}{
  morekeywords={color,background,margin,padding,font-size,font-family,border,width,height,display,position,float,clear,top,right,bottom,left},
  sensitive=true,
  morecomment=[l]{//},
  morecomment=[s]{/*}{*/},
  morestring=[b]
  morestring=[b]',
}
\lstdefinelanguage{js}{
  keywords={break, case, catch, continue, debugger, default, delete, do, else, false, finally, for, function, if, in, instanceof, new, null, return, switch, this, throw, true, try, typeof, var, void, while, with},
  morecomment=[l]{//},
  morecomment=[s]{/*}{*/},
  morestring=[b]',
  morestring=[b]",
  ndkeywords={class, export, boolean, throw, implements, import, this},
  keywordstyle=\color{blue}\bfseries,
  ndkeywordstyle=\color{darkgray}\bfseries,
  identifierstyle=\color{black},
  commentstyle=\color{purple}\ttfamily,
  stringstyle=\color{red}\ttfamily,
  sensitive=true
}

%%%



%%%set the font (only works with LuaLaTeX or XeLaTeX.)
%\usepackage{fontspec}
%\setmainfont{Georgia}
%\usepackage[math-style=ISO]{unicode-math} %trying to match georgia font for math
%\setmathfont{STIX Two Math}
%%%




%%%better underline (\ul or \bful and \sout for strikeout)
\usepackage{ulem} 
\renewcommand{\ULdepth}{1.5pt}

\usepackage{contour} %for not overlapping uline with low letters
\contourlength{0.5pt}

\newcommand{\ul}[1]{\uline{\phantom{#1}}\llap{\contour{white}{#1}}}  %normal uline
\newcommand{\bful}[1]{\def\ULthickness{0.7pt}\def\ULdepth{1.7pt}\uline{\phantom{#1}}\llap{\contour{white}{#1}}} %for boldface
\newcommand\reduline{\bgroup\markoverwith{\textcolor{red}{\rule[-0.5ex]{5pt}{0.4pt}}}\ULon} %for red text
%%%  




%%% used to bring in tex from other subfiles (could be subsections or python output)
\usepackage{blindtext} 
\usepackage{subfiles}
%%%

%
% Basic Document Settings
%

\topmargin=-0.45in
\evensidemargin=0in
\oddsidemargin=0in
\textwidth=6.5in
\textheight=9.0in
\headsep=0.25in

\linespread{1.1}

\pagestyle{fancy}
\lhead{\hmwkAuthorNameShort}
\chead{\hmwkClass : \hmwkTitle}
\rhead{\hmwkDueDate}
\lfoot{\lastxmark}
\cfoot{\thepage}

\renewcommand\headrulewidth{0.4pt}
\renewcommand\footrulewidth{0.4pt}

\setlength\parindent{0pt}

%
% Create Question Sections
%

\newcommand{\enterQuestionHeader}[1]{
   \nobreak\extramarks{}{Question \arabic{#1} continued on next page\ldots}\nobreak{}
   \nobreak\extramarks{Question \arabic{#1} (continued)}{Question \arabic{#1} continued on next page\ldots}\nobreak{}
}

\newcommand{\exitQuestionHeader}[1]{
   \nobreak\extramarks{Question \arabic{#1} (continued)}{Question \arabic{#1} continued on next page\ldots}\nobreak{}
   \stepcounter{#1}
   \nobreak\extramarks{Question \arabic{#1}}{}\nobreak{}
}

\setcounter{secnumdepth}{0}
\newcounter{partCounter}
\newcounter{homeworkQuestionCounter}
\setcounter{homeworkQuestionCounter}{1}
\nobreak\extramarks{Question \arabic{homeworkQuestionCounter}}{}\nobreak{}

%
% Homework Question Environment
%
% This environment takes an optional argument. When given, it will adjust the
% Question counter. This is useful for when the Questions given for your
% assignment aren't sequential. See the last 3 Questions of this template for an
% example.
%
\newenvironment{homeworkQuestion}[1][-1]{
   \ifnum#1>0
      \setcounter{homeworkQuestionCounter}{#1}
   \fi
   \section{Question \arabic{homeworkQuestionCounter}}
   \setcounter{partCounter}{1}
   \enterQuestionHeader{homeworkQuestionCounter}
}{
   \exitQuestionHeader{homeworkQuestionCounter}
}








%
% Title Page
%
\title{
   \vspace{2in}
   \textmd{\textbf{\hmwkClass:\ \hmwkTitle}}\\
   \normalsize\vspace{0.1in}\small{\hmwkDueDate\ }\\
   \vspace{0.1in}\large{\textit{\hmwkClassInstructor\ }}\\
%   \vspace{0.1 in}\large{\textit{\hmwkClassTA\ }} %uncomment if the T.A. is relevant
   \vspace{3in}
}

\author{\hmwkAuthorName}
\date{}

\renewcommand{\part}[1]{\textbf{\large Part \Alph{partCounter}}\stepcounter{partCounter}}

%
% Various Helper Commands
%

% Useful for algorithms
\newcommand{\alg}[1]{\textsc{\bfseries \footnotesize #1}}

% For derivatives
\newcommand{\deriv}[1]{\frac{\mathrm{d}}{\mathrm{d}x} (#1)}

% For partial derivatives
\newcommand{\pderiv}[2]{\frac{\partial}{\partial #1} (#2)}

% Integral dx
\newcommand{\dx}{\mathrm{d}x}

% Alias for the Answer section header
\definecolor{darkblue}{HTML}{00008B}
\newenvironment{answer}{


   \smallskip
   \color{darkblue}

}

% Probability commands: Expectation, Variance, Covariance, Bias
\newcommand{\E}{\mathrm{E}}
\newcommand{\Var}{\mathrm{Var}}
\newcommand{\Cov}{\mathrm{Cov}}
\newcommand{\Bias}{\mathrm{Bias}}

% custom macros
\newcommand{\img}[1]{\begin{center}\includegraphics[width=5in]{#1}\end{center}}
\newcommand{\codehtml}[1]{\lstinputlisting[language=HTML]{#1}}
\newcommand{\codecss}[1]{\lstinputlisting[language=CSS]{#1}}

